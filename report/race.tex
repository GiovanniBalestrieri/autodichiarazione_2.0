{\color{PineGreen}\section{The race to new methods}}

In this section we quickly discuss the methods used by other countries to improve contact tracing.

{\color{PineGreen}\subsection{Main idea}}

Researchers and technologists all around the world are developing several mobile based solutions that alert users when they have come into contact with someone with coronavirus. Proximity or location data are used to notify someone who has recently been near to an infected individual, so that they can then take preventive action such as self-isolating.\\
\\
These solutions could safely help transition out of national lockdowns and relax social distancing rules while defending against a second wave of infections. Medical researchers and bioethicists at the University of Oxford found that digital contact tracing had the potential to achieve epidemic control if used by enough people.
\\
{\color{PineGreen}\subsection{Bluetooth based methods}}
One of the most popular formats to emerge have been bluetooth based apps. By taking advantage of th low energy protocol, bluetooth identifiers and signal strength from other nearby phones they keep a record of them for a set period of time. This approach allows both outdoor and indoor tracking and is considered by privacy advocates as the least intrusive form of mobile tracking.
Several solutions have been released under free software or less restrictive open source licences. Other proprietary solutions are spreading as well. 
\\
{\color{PineGreen}\subsection{Gps Alternatives}}

An alternative solution could be GPS location tracking technology. Teams at MIT are exploring this solution (Safe Paths) on top of their bluetooth offering. There is emerging evidence that coronavirus can be transferred via surfaces even after a period of time suggesting it may be better to monitor where someone went rather than with whom they crossed paths.\\
\\
But GPS location data is harder to anonymise and researchers are still looking for ways to better encrypt data.
\\
{\color{PineGreen}\subsection{Related problems}}
The main concerns are related to privacy and the possibility of data leakage by the involved actors in the logic.\\
Related information:
\begin{itemize}
\item Beijing appeared to share citizen's data with the police.
\item South Korea has broadcast the personal information of infected people when alerting others who may have been exposed to them.
\item In Israel, security services are controversially tapping data collected by the country's mobile phone operators.
\end{itemize}

{\color{PineGreen}\subsection{Indentified proprietary solutions}}
\begin{itemize}
\item \textit{Pan-European Privacy-Preserving Proximity Tracing}\cite{bib4} claims to interrupt new chains of SARS-CoV-2 transmission rapidly and effectively by informing potentially exposed people. They enable tracing of infection chains across national borders. 
\item \textit{Covid19 Contact Alert} Combining NFC and Blockchain technology to monitor contact moments and alert people. German technology company MYNXG\cite{bib5} introduces a blockchain technology to enable privacy compliant pandemic tracking on regular smartphones. People can download a special app on their smartphone and use any existing NFC to connect their smartphone to the MYNXG blockchain. This technology provides privacy when monitoring individual movements or alerting people if there is an infection in their contact chain. 
\end{itemize}


